\documentclass{ITESO-Prereporte}
\usepackage[utf8]{inputenc}
\usepackage[spanish]{babel}
\usepackage[sfdefault]{carlito}
\usepackage{csquotes}
\usepackage[style=apa, backend=biber]{biblatex}
\usepackage{lipsum}
\usepackage{tikz}

\addbibresource{biblio.bib}
\addto\captionsspanish{\renewcommand*{\contentsname}{\color{azuliteso1}\large\bfseries Tabla de contenidos}} % Cambiar el título de \tableofcontents

\dpto{Departamento de Procesos Tecnológicos e Industriales}{DPTI}
\numlab{NN}
\profesor{El Nombre del Profe}
\periodo{Otoño}{2022}

\title{Título del reporte de laboratorio, con incluso en \textit{itálicas} para latín}
\author{Roberto Olvera-Hernández}
\authexp{ib721045}
\date{\today}

\mylab{El laboratorio}{Código}

\begin{document}

\maketitlepage
\nocite{*} % Solo para imprimir bibliografía sin escribir (\parencite{}).

%%% Inicio del documento
\newpage{\huge\bfseries\color{gray} Pre-reporte}

\question{Esta es la primera pregunta del documento}
\lipsum[1]

\question{Y esta es otra pregunta}
\lipsum[2]

\newpage{\huge\bfseries\color{gray} Diagrama de flujo}

\tikzstyle{process} = [rectangle, very thick, rounded corners, minimum width=3cm, minimum height=1cm,text centered, text width=3cm, draw=black, fill=white]
\tikzstyle{decision} = [diamond, aspect=1.5, text centered, text width=2.5cm, draw=black, fill=yellow!25]

\section{Construcción del puente salino}

\begin{center}
  \begin{tikzpicture}
      % Proceso opcional
      \node (op01) [process] {Pesar 48g de agarosa y 16g de KCl};
      \node (op02) [process,  below of=op01, yshift=-1.5cm] {Diluir en 400mL en un vaso de precipitado};
      \node (op03) [process,  right of=op02, xshift=3cm] {Cortar tubos de plástico de 5cm de largo};
      \node (op04) [process,  right of=op03, xshift=3cm] {Rellenar con agarosa previamente preparada};
      \node (op05) [process,  right of=op04, xshift=3cm] {Dejar reposar $\times$ 10min};

      % Conexiones
      \draw [-stealth, very thick] (op01) -- (op02);
      \draw [-stealth, very thick] (op02) -- (op03);
      \draw [-stealth, very thick] (op03) -- (op04);
      \draw [-stealth, very thick] (op04) -- (op05);
  \end{tikzpicture}
\end{center} % Diagrama de flujo 01


\newpage\printbibliography[title=Referencias citadas]



\end{document}